\documentclass{article}
\usepackage[english]{babel}
\usepackage{geometry,alltt}
\geometry{letterpaper}

%%%%%%%%%% Start TeXmacs macros
\newcommand{\tmcodeinline}[2][]{{\ttfamily{#2}}}
\newcommand{\tmstrong}[1]{\textbf{#1}}
%%%%%%%%%% End TeXmacs macros

\begin{document}

1.2 In the output from the program in {\tmstrong{Figure 1.6}}, what happened
to the processes with process IDs 852 and 853?

The two processes with IDs 852 and 853 sprang into existence in between 851
and 854 were created.

\

1.3 In Section 1.7, the argument to \tmcodeinline[cpp]{perror} is defined
with the ISO C attribute \tmcodeinline[cpp]{const}, whereas the integer
argument to \tmcodeinline[cpp]{strerror} isn't defined with this attribute.
Why?

\tmcodeinline[cpp]{Perror}'s argument is a pointer. Without
\tmcodeinline[cpp]{const} the function could change the content to which the
pointer points. While \tmcodeinline[cpp]{strerror} takes integer as argument.
In no way it could change the caller's stack.

1.4 If the calendar time is stored as a \tmcodeinline[cpp]{signed} 32-bit
integer, in which year will it overflow? How can we extend the overflow point?
Are these strategies compatible with existing applications?

Get the accurate date using ruby:
\begin{alltt}
irb(main):012:0> Time.at(("1"*31).to_i(2)).getutc
=> 2038-01-19 03:14:07 UTC
\end{alltt}
Using data types with larger width. It is not compatible with existing
applications.

1.5 If the process time is stored as a signed 32-bit integer, and if the
system counts 100 ticks per second, after how many days will the value
overflow?

Get result:
\begin{alltt}
irb(main):012:0> Time.at(("1"*31).to_i(2)).getutc
=> 2038-01-19 03:14:07 UTC
\end{alltt}

\end{document}
